%! Author = t
%! Date = 3/26/24

% Preamble
\documentclass[12pt]{article}

% Packages
\usepackage{amsmath}
\usepackage{enumitem}
\usepackage{titlesec}
\usepackage[table]{xcolor}
\usepackage{subcaption,xcolor,lipsum}

% Formatting
\titleformat{\section}[hang]{\fontsize{20}{24}\selectfont\filcenter}{\Roman{section}}{1em}{}

\title{Networks Assignment\\ Chapter 3}
\author{Braiko Timofei\\1820243088}
\date{\today}
\maketitle
\newpage

\begin{document}
    \section{Task}\label{sec:task-1}

    \subsection{Problem}
        A bit stream 10011101 is transmitted using the standard CRC method described in the text.
        The generator polynomial is $x^3$ + 1.
        Show the actual bit string transmitted.
        Suppose that the third bit from the left is inverted during transmission.
        Show that this error is detected at the receiver’s end.
        Give an example of bit errors in the bit string transmitted that will not be
        detected by the receiver.

        \subsection{Encoding}

        The generator polynomial is $x^3+1 \longrightarrow$ 1001,
        so we need to append 3 zeros at the end of data and calculate the remainder.

        Thus:
        %    Encoding
        \[\arraycolsep=0.3em
        \begin{array}{rrrrrrrrrrr@{\,}r|l}
            1 & 0 & 0 & 1 & 1 & 1 & 0 & 1 & 0 & 0 & 0 & & \,1001 \\
            \cline{13-10}
            1 & 0 & 0 & 1 &   &   &   &   &   &   &   & & \,10001100     \\
            \cline{1-4}
            & 0 & 0 & 0 & 1 \\
            & 0 & 0 & 0 & 0 \\
            \cline{2-5}
            & & 0 & 0 & 1 & 1 \\
            & & 0 & 0 & 0 & 0 \\
            \cline{3-6}
            & & & 0 & 1 & 1 & 0 \\
            & & & 0 & 0 & 0 & 0 \\
            \cline{4-7}
            & & & & 1 & 1 & 0 & 1 \\
            & & & & 1 & 0 & 0 & 1 \\
            \cline{5-8}
            & & & & & 1 & 0 & 0 & 0 \\
            & & & & & 1 & 0 & 0 & 1 \\
            \cline{6-9}
            & & & & & & 0 & 0 & 1 & 0 \\
            & & & & & & 0 & 0 & 0 & 0 \\
            \cline{7-10}
            & & & & & & & 0 & 1 & 0 & 0 \\
            & & & & & & & 0 & 0 & 0 & 0 \\
            \cline{8-11}
            & & & & & & & & 1 & 0 & 0
        \end{array} \]

        We obtain the 100 remainder.
        Therefore, the actual transmitted data is:
        \begin{center}
            \boxed{ \underbrace{10011101}_{data}\underbrace{100}_{remainder} }
        \end{center}

        \subsection{Error Detecting}
        We consider that third bit from left side is inverted, so receiver gets 10111101100 message instead.
        The decoding will be the following:

        \[\arraycolsep=0.3em
        \begin{array}{rrrrrrrrrrr@{\,}r|l}
            1 & 0 & 1 & 1 & 1 & 1 & 0 & 1 & 1 & 0 & 0 & & \,1001 \\
            \cline{13-10}
            1 & 0 & 0 & 1 &   &   &   &   &   &   &   & & \,10101000     \\
            \cline{1-4}
            & 0 & 1 & 0 & 1 \\
            & 0 & 0 & 0 & 0 \\
            \cline{2-5}
            & & 1 & 0 & 1 & 1 \\
            & & 1 & 0 & 0 & 1 \\
            \cline{3-6}
            & & & 0 & 1 & 0 & 0 \\
            & & & 0 & 0 & 0 & 0 \\
            \cline{4-7}
            & & & & 1 & 0 & 0 & 1 \\
            & & & & 1 & 0 & 0 & 1 \\
            \cline{5-8}
            & & & & & 0 & 0 & 0 & 1 \\
            & & & & & 0 & 0 & 0 & 0 \\
            \cline{6-9}
            & & & & & & 0 & 0 & 1 & 0 \\
            & & & & & & 0 & 0 & 0 & 0 \\
            \cline{7-10}
            & & & & & & & 0 & 1 & 0 & 0 \\
            & & & & & & & 0 & 0 & 0 & 0 \\
            \cline{8-11}
            & & & & & & & & 1 & 0 & 0
        \end{array}
        \]

        As the remainder is not equal to zero, the error will be detected.
    \newpage


    \section{Task}\label{sec:task-2}

    \subsection{Problem}
        A channel has a bit rate of 4 kbps and a propagation delay of 20 msec.
        For what frame sizes does stop-and-wait give an efficiency of at least 50 $\%$?

        \subsection{Solution}
        Let assume that F is frame size.
        The formula of channel efficiency is the following:

        \begin{equation}
            U = \frac{T_d}{T_d + 2 \times P_d}
        \end{equation}

        As we are given a propagation delay: $20 msec = 0.02sec$, transmission delay: $F/(4\times10^3)sec$,
        and channel efficiency has to be at least 50$\%$ we obtain the following inequality:

        \begin{subequations}
            \begin{equation}
                0.5 \leq \frac{F/(4\times10^3)}{F/(4\times10^3) + 2 \times 0.02} = \frac{F}{F + 0.16 \times 10^3}
            \end{equation}

            \begin{equation}
                0.5(F + 160) \leq F
            \end{equation}

            \begin{equation}
                F \geq 160
            \end{equation}
        \end{subequations}

        Hence, the frame size should be at least 160 bit.

        \paragraph{Answer:} 160 bit
    \newpage


    \section{Task}\label{sec:task-3}
    \subsection{Problem}
        Suppose you are designing a sliding window protocol for a 1-Mbps point-to-point link to the
        stationary satellite evolving around the Earth at $3\times10^4$ km altitude.
        Assuming that each frame carries 1 kB of data,
        what is the minimum number of bits you need for the sequence number in the following cases?
        Assume the speed of light is $3\times10^8$ meters per second.

        \begin{enumerate}[label=(\alph*)]
            \item Receive Window Size = 1.
            \item Receive Window Size = Send Window Size
        \end{enumerate}

    \subsection{Solution}

        For the maximum sender window size we obtain the following:
            \begin{gather}
                SWS = \frac{2 \times T_d}{P_d} = \\
                = \frac{2 \times (3\times10^7)/(3\times10^8)}{(8\times10^3)/10^6} \\
                = \frac{10^3}{4\times10} = 25
            \end{gather}
        \paragraph{A)} As RWS = 1, we need 26 sequence numbers per frame.\\
                 Therefore, we need $\lceil \log_{2}(26) \rceil = 5$ bits for sequence numbers.
        \paragraph{B)} As RWS = SWS, we need 50 sequence numbers per frame.\\
                 Therefore, we need $\lceil \log_{2}(50) \rceil = 6$ bits for sequence numbers.

        \paragraph{Answers:} A) 5 bits\\ \hspace*{62} B) 6 bits

    \newpage


    \section{Task}\label{sec:task-4}
    \subsection{Problem}
    Suppose that we run the sliding window algorithm with SWS = 5 and RWS = 3,
        and no out-of-order arrivals
        \begin{enumerate}[label=(\alph*)]
            \item Find the smallest value for MaxSeqNum.
            You may assume that it suffices to find the smallest MaxSeqNum such
            that if DATA[MaxSeqNum] is in the receive window, then DATA[0] can no longer arrive.
            \item Give an example showing that MaxSeqNum − 1 is not sufficient.
            \item State a general rule for the minimum MaxSeqNum in terms of SWS and RWS.
        \end{enumerate}


    \subsection{Solution}
        \paragraph{A)} MaxSeqNum = RWS + SWS = 8.
                At provided picture depicts 8 frames that are transmitted.
                Green frames the receiver just get from the sender, while the blue one are sending.
                The minimum required sequential number is 8 need to avoid misconsumption.
        \begin{table}[h]
                \centering
                    \begin{tabular}{|c|c|c|c|c|c|c|c|c|c|c|}
                        \hline
                        \dots & \cellcolor{green!30}3 & \cellcolor{green!30}4
                        & \cellcolor{green!30}5 & \cellcolor{blue!30}6 & \cellcolor{blue!30}7
                        & \cellcolor{blue!30}8 & \cellcolor{blue!30}9 & \cellcolor{blue!30}10 & \dots \\
                        \hline
                    \end{tabular}
                \caption{Green - Reciever Window, Blue - Receiver Window}
                \label{tab:my_table}
        \end{table}

        \paragraph{B)} If we have MaxSeqNum = RWS + SWS - 1 = 7, the receiver might wait to get the Data[7] packet.
        However, ACK[0] didn't arrive to the sender.
        Thus, raises timeout and sender sends Data[0] again.
        As we have MaxSeqNum = 7, the receiver is not able to distinguish them at all.
    Dividing 7 and 0 by module such MaxSeqNum, receiver cannot distinguish them.
        \paragraph{C)} The general rule for MaxSeqNum have to be:
        \begin{center}
               $\bf MaxSeqNum \geq RWS + SWS$
        \end{center}

\end{document}