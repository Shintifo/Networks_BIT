%! Author = t
%! Date = 5/24/24

% Preamble
\documentclass[11pt]{article}

% Packages
\usepackage{amsmath}
\usepackage{titlesec}
\usepackage{enumitem}


\titleformat{\section}[hang]{\fontsize{20}{24}\selectfont\filcenter}{\Roman{section}}{1em}{}

\title{Networks Assignment\\ Chapter 7}
\author{Braiko Timofei\\1820243088}
\date{\today}
\maketitle
\newpage

% Document
\begin{document}
    \section{Task}\label{sec:task_1}
    File Transfer Protocol (FTP) uses \_\_\_\_\_\_\_ for control connection and \_\_\_\_\_\_\_ for data connection.
    \begin{enumerate}[label=\Alph*.]
        \item Stateless connection. Non persistent TCP connection
        \item Persistent TCP connection, non-persistent TCP connection
        \item Non-persistent TCP connection, persistent TCP connection
        \item FTP is a connection less protocol
    \end{enumerate}

    \subsection{Solution}
    File Transfer Protocol (FTP) uses Persistent TCP connection for control connection and non-persistent TCP connection for data connection.

    \paragraph{Answer:} B
    \newpage


    \section{Task}\label{sec:task_2}
    Consider different activities related to email. \\
    s1: Send an email from a mail client to a mail server \\
    s2: Download an email from mailbox server to a mail client \\
    s3: Checking email in a web browser \_\_\_\_\_\_ is the application level protocol used in each activity. \\

    \begin{enumerate}[label=\textbf{\Alph*.}]
        \item s1: HTTP , s2: SMTP , s3: POP
        \item s1: SMTP , s2: FTP , s3: HTTP
        \item s1: SMTP , s2: POP , s3: HTTP
        \item s1: POP , s2: SMTP , s3: IMAP
    \end{enumerate}

    \paragraph{Answer:}  C) SMTP, POP, HTTP
    \newpage


    \section{Task}\label{sec:task_3}
    Suppose within your Web browser you click on a link to obtain a Web page.
    The IP address for the associated URL is not cached in your local host,
    so a DNS lookup is necessary to obtain the IP address.
    Suppose that n DNS servers are visited before your host receives the IP address from DNS;
    the successive visits incur an RTT of $RTT_1$, ..., $RTT_n$.
    Further suppose that the Web page associated with the link contains exactly one object,
    consisting of a small amount of HTML text.
    Let RTT denote the RTT between the local host and the server containing the object.
    Assuming zero transmission time of the object,
    how much time elapses from when the client clicks on the link until the client receives the object?

    \subsection{Solution}
    To look the final IP address through multiple DNS servers takes:
    \begin{subequations}
        \begin{equation}
            \sum_{i=1}^{n} RTT_{i}
        \end{equation}
        Finally, to request the HTML object from the server takes RTT sec.
        Hence, the total time is:
        \begin{equation}
            \sum_{i=1}^{n} RTT_{i} + RTT
        \end{equation}
    \end{subequations}
    \newpage


    \section{Task}\label{sec:task_4}
    Referring to Problem 3, suppose the HTML file references eight very small objects on the same server.
    Neglecting transmission times, how much time elapses with
    \begin{enumerate}[label=(\alph*)]
        \item \textbf{Non-persistent HTTP with no parallel TCP connections?} \\
        I consider that host required firstly obtain the base file containing text.
        \begin{subequations}
            To get base file:
            \begin{equation}
                \sum_{i=1}^{n} RTT_{i} + RTT
            \end{equation}
            Per one object we need 4RRTs.
            Thus, in total:
            \begin{equation}
                \sum_{i=1}^{n} RTT_{i} + 33RTT
            \end{equation}
        \end{subequations}
        \item \textbf{Non-persistent HTTP with the browser configured for 5 parallel connections?} \\
        \begin{subequations}
            \begin{equation}
                \sum_{i=1}^{n} RTT_{i} + RTT + 4RTT(5 objects) + 4RTT(3 objects)
            \end{equation}
            \begin{equation}
                \sum_{i=1}^{n} RTT_{i} + 9RTT
            \end{equation}
        \end{subequations}
        \item \textbf{Persistent HTTP?} \\
        \begin{subequations}
            Per one object we need 2RTTs. Therefore, in total it requires:
            \begin{equation}
                \sum_{i=1}^{n} RTT_{i} + 17RTT
            \end{equation}
        \end{subequations}
    \end{enumerate}

\end{document}